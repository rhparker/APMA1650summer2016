
% Document settings
\documentclass[11pt]{article}
\usepackage[margin=1in]{geometry}
\usepackage[pdftex]{graphicx}
\usepackage{multirow}
\usepackage{setspace}
\pagestyle{plain}
\setlength\parindent{0pt}

\begin{document}

% Course information
\begin{tabular}{ l }
 \LARGE Math 351: Real Analysis \\\\
 \LARGE MWF 10:00-10:50 am \\\\
 \LARGE Barus and Holley Room 161
\end{tabular}
\vspace{10mm}

% Professor information
\begin{tabular}{ l }
   \large Instructor: Ross Parker \\
   \large Office Location: 182 George St, room 024 \\
   \large Office Hours: TTh 2 - 3:30 pm \\
   \large ross\_parker@brown.edu \\
\end{tabular}
\vspace{5mm}

\section*{Course Description}

Take a moment and think back to your calculus course. First you studied limits of functions. You looked at strange functions like $\sin{x} / x$ and learned, for example, that
\[
\lim_{x \rightarrow 0} \frac{\sin{x}}{x} = 1
\]
even though it looks like you are dividing zero by zero. Next you studied how functions change. You defined the instantaneous rate of change of a function -- the ``slope of a curve'' -- using a special limit called the derivative. You memorized arcane formulas such as the chain rule and the product rule to find the derivative of complicated functions. From there, you leapt to the seemingly unrelated problem of calculating the area under the curve, known as the integral. You learned that, almost by magic, the concepts of derivative and integral are intimately intertwined; maybe you came to think of them as opposites. Finally you looked at infinite sequences and series and perhaps thought to yourself, ``What does this have to do with anything?'' If this has kept bothering you, if you have been unable to sleep at night because you cannot not stop wondering how and why this all works, then this is course for you!
\\\\
Formally, real analysis is the study of the real number system and of functions of a real variable. Essentially, it is a rigorous study of the mathematics underlying calculus. We will start with basic properties of the real number system. What makes real numbers special? We will then take a brief digression into the realm of topology to discuss metric spaces, sets on which a notion of distance is defined. This leads to two abstract topological properties -- compactness and connectedness -- which will prove to be very useful. From there, we will cover sequences and series of real numbers (just like at the end of calculus!), and will formally define what is meant by  convergence. We will then discuss functions of real numbers. We will define the limit of a function and will relate properties of limits to properties of convergent sequences. We will then define continuity of a function and relate that to the topological properties we learned earlier. \\\\
Then it's on to more familiar territory. As in your calculus course, we start by defining the derivative, but this time we rigorously prove all the formulas and relations you have come to know and love. The next step is to define the Riemann integral as the limit of ``approximating with boxes''. We then prove the Fundamental Theorem of Calculus, the glue which connects the processes of differentiation and integration. We conclude the semester where we started by returning to sequences and series. This time, however, it's sequences and series of functions. These are much more complicated, but we are able to handle them with ease using all the tools we have developed over the course of the semester.
\\\\
Real analysis is a difficult subject. It is likely your first introduction to rigorous mathematical proof. Many of the concepts you will encounter are highly abstract and seem outright bizarre the first time you encounter them (they were for me too!) What this means practically is that you will need to spend a significant amount of time outside of class reading, reviewing material, and wrestling with these new ideas. My job is to make this as easy as possible, but at the end of the day there is no way to avoid some amount amount of frustration and perspiration. I promise you will find this rewarding, and I invite you to join me on this mathematical journey which will challenge you in new and exciting ways!

\section*{Course Objectives}
By the end of the course you should:
\begin{enumerate}
\item Have a solid understanding of the central definitions and theorems of real analysis as well as how the subject is built up from ``first principles''. (Knowledge)
\item Be able to construct examples illustrating these definitions and theorems. These can be pictures or more ``mathy'' things like functions. You will be asked to provide examples and counterexamples on the problem sets and exams. (Understanding)
\item Be able to write a rigorous, concise, and clear mathematical proof. You will demonstrate this by writing a complete, formal proof of a theorem from real analysis. (Communication, how to think like a mathematician)
\end{enumerate}

\section*{Nuts and Bolts}

\textbf {Class format:} This is a lecture-based course, with 50-minute lectures three times per week. There will be a one-hour problem session (to be scheduled at a mutually convenient time) during which students will present and discuss homework problems. Problems to be presented will be assigned to individual students one week in advance of the problem sessions. Although attendance at lectures is not strictly required, it is strongly encouraged, as the course will move very quickly, and each lecture will build on material from previous lectures . Lecture notes will be posted at the end of each week covering the material presented in lecture that week.
\\\\
\textbf {Homework: }There will be weekly problem sets. These will be assigned each Monday and will be due on the following Monday. You are encouraged to work together on assignments, but you must write up your own solutions. At the end of the semester, the lowest homework score will be dropped. \\\\
\textbf{Homework policy: }Late assignments generally will not be accepted. That being said, I understand that the unexpected does happen. If you have a serious situation in which you will believe you will be unable to complete your assignment on time, contact me directly and we will arrange something.\\\\
\textbf{Final project: }You will write a complete, formal proof of a theorem from real analysis. I will give you a small list of theorems to select from, with the option of choosing one for yourself (subject to my approval). Over the course of the semester, you will present several drafts to me, with the final draft due on the day of the final exam. Deadlines for each draft will be posted on the course website. I will meet with you individually after each draft to discuss appropriate revisions and ideas for the next draft.\\\\
\textbf{Communication: }Email is the best way to reach me. During the week, I will try to respond within 24 hours. Email responses may be slower on the weekends, but I will try to reply by Sunday evening. For complicated questions, I may ask you to talk with me after class or come to my office hours.
\\\\
\textbf {\large Text:} To be decided \\\\
\textbf {Prerequisites:} A course in multivariable calculus and a course in linear algebra. \\\\
\textbf {Grade Distribution:} \\
\hspace*{40mm}
\begin{tabular}{ l l }
Problem Sets & 20\% \\
Midterm Exam  & 30\% \\
Final Project & 20\% \\
Final Exam  & 30\%
\end{tabular} \\\\

\section*{Academic Honesty Policy}
Norms regarding the quality and originality of academic work are often much more stringent and demanding in college than they are in high school. All Brown students are responsible for understanding and following Brown’s academic code, which is described below. \\

Academic achievement is ordinarily evaluated on the basis of work that a student produces independently. Students who submit academic work that uses others’ ideas, words, research, or images without proper attribution and documentation are in violation of the academic code. Infringement of the academic code entails penalties ranging from reprimand to suspension, dismissal, or expulsion from the University. \\

Brown students are expected to tell the truth. Misrepresentations of facts, significant omissions, or falsifications in any connection with the academic process (including change of course permits, the academic transcript, or applications for graduate training or employment) violate the code, and students are penalized accordingly. This policy also applies to Brown alums, insofar as it relates to Brown transcripts and other records of work at Brown. \\

Misunderstanding the code is not an excuse for dishonest work. Students who are unsure about any point of Brown’s academic code should consult their courses instructors or an academic dean, who will be happy to explain the policy.

\section*{Academic Support}
Brown University is committed to full inclusion of all students.  Please inform me if you have a disability or other condition that might require accommodations or modification of any of these course procedures. You may speak with me after class or during office hours. For more information contact Student and Employee Accessibility Services at 401-863-9588 or SEAS@brown.edu.

\end{document}