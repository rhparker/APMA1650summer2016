% \documentclass{book}

\documentclass[12pt]{article}
\usepackage[pdfborder={0 0 0.5 [3 2]}]{hyperref}%
\usepackage[left=1in,right=1in,top=1in,bottom=1in]{geometry}%
\usepackage[shortalphabetic]{amsrefs}%
\usepackage{amsmath}
\usepackage{enumerate}
\usepackage{enumitem}
\usepackage{amssymb}                
\usepackage{amsmath}                
\usepackage{amsfonts}
\usepackage{amsthm}
\usepackage{bbm}
\usepackage[table,xcdraw]{xcolor}
\usepackage{tikz}
\usepackage{float}
\usepackage{booktabs}
\usepackage{svg}
\usepackage{mathtools}
\usepackage{cool}
\usepackage{url}
\usepackage{graphicx,epsfig}
\usepackage{makecell}
\usepackage{array}

\def\noi{\noindent}
\def\T{{\mathbb T}}
\def\R{{\mathbb R}}
\def\N{{\mathbb N}}
\def\C{{\mathbb C}}
\def\Z{{\mathbb Z}}
\def\P{{\mathbb P}}
\def\E{{\mathbb E}}
\def\Q{\mathbb{Q}}
\def\ind{{\mathbb I}}

\graphicspath{ {images/} }

\begin{document}

\title{}
\author{\vspace{-10ex} }

\begin{center}
{\LARGE APMA 1650 -- Review Session 2}\\
\vspace{5mm}
{\large Monday, July 25, 2016}\\
% \vspace{5mm}
\end{center}

\begin{enumerate}

\item The following series of questions refers to the SAT, everyone's favorite standardized test.
\begin{enumerate}
\item The mean score on the SAT math section is 511. What is an upper bound on the probability that a student scores over 700? 

\item The standard deviation is 120 of the SAT math section is 120. Can you get a better upper bound on the probability that a student scores over 700?

\item The College Board makes great effort to ensure that SAT scores are roughly normally distributed. Assuming this is the case, what is the probability that a student scores over 700 on the SAT math section?
\end{enumerate}

\item You are a barista at a local coffee shop. The average number of customers per hour who enter your shop is 10. Assume customers arrive one-at-a-time and their arrivals are independent from each other.
\begin{enumerate}
\item What is the probability that fewer than 3 customers will enter your coffee shop in one hour?

\item What is the average time between the arrival of two customers?

\item What is the probability that there will be an interval of 10 minutes or more between the arrival of one customer and the 
\end{enumerate}

\item Let $X$ and $Y$ have a joint density function given by
\[
f(x,y) = \begin{cases}
c x & 0 \leq y \leq x \leq 3 \\
0 & otherwise
\end{cases}
\]
\begin{enumerate}
\item Find the value of $c$ such that this is valid joint density function.

\item Find the marginal densities of $X$ and $Y$.

\item Find the expected values of $X$ and $Y$.

\item Find the conditional density of $Y$ given $X = x$.

\item Find $\P(Y \leq  1/2 | X = 1)$

\item Find the conditional expected value $\E(Y|X = x)$.
\end{enumerate}

\item Let $X$ and $Y$ be random variables with joint density given by
\[
f(x, y) = \begin{cases}
6(1-y) & 0 \leq x \leq y \leq 1\\
0 & \text{otherwise}
\end{cases}
\]
Find the covariance of $X$ and $Y$. Are $X$ and $Y$ independent?

\item A forester studying the effects of fertilization on pine forests is interested in estimating the average basal area of pine trees (basal area is the area of a given section of land that is occupied by the cross-section of tree trunks at their base). She has discovered that these measurements (in square inches) are normally distributed with standard deviation of 4 square inches.
\begin{enumerate}
\item If she samples $n = 9$ trees, what is the probability that the sample mean will be within 2 square inches of the population mean.

\item If she would like the sample mean to be within 1 square inch of the population mean with probability 0.90, how many trees must she measure in order to ensure this degree of accuracy?
\end{enumerate}

\item There are 3 boxes on a table, containing $0, \theta$, and $\theta+1$ jellybeans.  Each of $n$ people opens a box uniformly at random and takes the amount of jellybeans in the box. (The boxes are reset after each person takes their turn.) Let $X_1, \dots, X_n$ be the number of jellybeans taken by each of the $n$ people.
\begin{enumerate}
\item Show $\hat{\theta} = \bar{X} = (1/n) \sum_{i=1}^n X_i$ is a biased estimator for $\theta$. 

\item Based on the above, how can we modify $\hat{\theta}$ to convert it into an unbiased estimator.
\end{enumerate}

\item Suppose that the number of minutes late a RIPTA bus arrives is uniformly distributed on an interval [0, 12]. Suppose you measured the delay of a RIPTA bus 64 times and computed the sample mean $\bar{Y}$. What is the probability that the sample mean is between 5 and 7 minutes?

\end{enumerate}
\end{document}