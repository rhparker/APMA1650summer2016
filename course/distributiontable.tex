% \documentclass{book}

\documentclass[12pt]{article}
\usepackage[pdfborder={0 0 0.5 [3 2]}]{hyperref}%
\usepackage[left=1in,right=1in,top=1in,bottom=1in]{geometry}%
\usepackage[shortalphabetic]{amsrefs}%
\usepackage{amsmath}
\usepackage{enumerate}
\usepackage{enumitem}
\usepackage{amssymb}                
\usepackage{amsmath}                
\usepackage{amsfonts}
\usepackage{amsthm}
\usepackage{bbm}
\usepackage[table,xcdraw]{xcolor}
\usepackage{tikz}
\usepackage{float}
\usepackage{booktabs}
\usepackage{svg}
\usepackage{mathtools}
\usepackage{cool}
\usepackage{url}
\usepackage{graphicx,epsfig}
\usepackage{makecell}
\usepackage{array}
\setlength\extrarowheight{25pt}

\begin{document}

\begin{figure}[H]
\caption{Discrete Distributions}
\begin{tabular}{l c c c c}
\hline
Distribution & Parameters & Probability Mass Function (pmf) & Mean & Variance \\
\hline
Binomial & $n, p$ & \makecell{ $\displaystyle p(r) = \binom{n}{r}p^r(1-p)^{n-r}$\\$ \displaystyle r = 0, 1, \dots, n$} & $np$ & $np(1-p)$ \\
Geometric & $p$ & \makecell{ $\displaystyle p(r) = p(1-p)^{r-1}$ \\ $r = 1, 2, \dots$} & $ \displaystyle \frac{1}{p}$ & $\displaystyle \frac{1 - p}{p^2}$ \\
Poisson & $\lambda$ & \makecell{ $\displaystyle p(r) = \frac{\lambda^r e^{-\lambda}}{r!}$ \\ $r = 0, 1, 2, \dots$ } & $\lambda$ & $\lambda$ \\
\end{tabular}
\end{figure}

\vspace{2cm}

\begin{figure}[H]
\caption{Continuous Distributions}
\begin{tabular}{l c c c c}
\hline
Distribution & Parameters & Probability Density Function (pdf) & Mean & Variance \\
\hline
Uniform & $a, b$ & \makecell{ $\displaystyle f(y) = \frac{1}{b-a}$ \\ $a \leq y \leq b$ }& $\displaystyle \frac{a + b}{2}$ & $\displaystyle \frac{(b - a)^2}{12}$ \\
Exponential & $\lambda$ & \makecell{ $\displaystyle f(y) = \lambda e^{-\lambda y}$ \\ $0 \leq y < \infty$} & $\displaystyle \frac{1}{\lambda}$ & $\displaystyle \frac{1}{\lambda^2}$ \\
Normal & $\mu, \sigma$ & $\displaystyle f(y) = \frac{1}{\sqrt{2 \pi}\sigma}e^{- \frac{(y - \mu)^2}{2 \sigma^2}}$ & $\mu$ & $\sigma^2$ \\
Standard Normal & none & $\displaystyle f(y) = \frac{1}{\sqrt{2 \pi}}e^{- \frac{y^2}{2}}$ & $0$ & $1$ \\
\end{tabular}
\end{figure}

\end{document}
