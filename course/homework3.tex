% \documentclass{book}

\documentclass[12pt]{article}
\usepackage[pdfborder={0 0 0.5 [3 2]}]{hyperref}%
\usepackage[left=1in,right=1in,top=1in,bottom=1in]{geometry}%
\usepackage[shortalphabetic]{amsrefs}%
\usepackage{amsmath}
\usepackage{enumerate}
\usepackage{enumitem}
\usepackage{amssymb}                
\usepackage{amsmath}                
\usepackage{amsfonts}
\usepackage{amsthm}
\usepackage{bbm}
\usepackage[table,xcdraw]{xcolor}
\usepackage{tikz}
\usepackage{float}
\usepackage{booktabs}
\usepackage{svg}
\usepackage{mathtools}
\usepackage{cool}
\usepackage{url}
\usepackage{graphicx,epsfig}
\usepackage{framed}
\usepackage{hyperref}  

\graphicspath{ {images/} }

\begin{document}

\title{}
\author{\vspace{-10ex} }

\begin{center}
{\LARGE APMA 1650 -- Homework 11}\\
\vspace{5mm}
{\large Due Monday, July 11, 2016}\\
\vspace{5mm}
Homework is due during class or by 3:45 pm in the homework drop box in 182 George St.\\
Show all of your work used in deriving your solutions.
\end{center}

\begin{enumerate}

\item Two people are playing a game. They take turns rolling a standard, fair six-sided die. The game ends when one player rolls a 6. The player who rolls the 6 is the winner of the game. A ``round'' of the game is defined as a single die roll.
\begin{enumerate}
\item What is the probability that the player who goes first wins?
\item What is the expected value and variance for the number of rounds in the game?
\item What is the probability that the game lasts at least 4 rounds?
\item Given the game has lasted four rounds, what is the probability that the game lasts at least 8 rounds?
\end{enumerate}

\item A single cell will die with probability $p$ or split into two cells with probability $1 − p$, producing a second generation of cells. Each cell in the second generation (if there are any) will die or split into two with the same probabilities as the initial cell. You start with a single cell.
\begin{enumerate}
\item Find the probability mass function (pmf) for the number of cells in the third generation.
\item What is the expected value of the number of cells in the third generation?
\end{enumerate}

\item When you are not studying for APMA 1650, you work part-time as a barista at a local coffee shop. Customers arrive at your coffee shop at an average rate of 10 customers per hour.
\begin{enumerate}
\item Model this problem with an appropriate probability distribution. What is the probability that fewer than 5 customers arrive in a fixed, one-hour period?
\item What is the probability that 5 or more customers arrive in a fixed, one-hour period?\\

Suppose it takes 4 minutes to serve one customer. The total service time is the number of minutes during a fixed, one-hour period which are spent serving customers.
\item What is the average total service time?
\item What is the variance of the total service time?
\end{enumerate}

\item A particular flight can only fit 200 people, but tickets were sold to 205 people. Suppose each ticket holder has a 0.05 probabiltiy of not showing up for the flight.
\begin{enumerate}
\item What is the probability that the flight will be overbooked? Give an approximate numerical answer for this.
\item What is the expected number of people that show up for the flight?
\item What is the variance of the number of people who show up for the flight?
\end{enumerate}

\item You reprise your role as the quality control manager for the Acme Widget Company. You have found that in every box of 100 widgets there is on average 1 defective widget.
\begin{enumerate}
\item Model this problem with an appropriate probability distribution. What is the probability that a box of 100 widgets contains 2 or fewer defective widgets?
\item Approximating this with a Poisson distribution, find the probaility that you have 2 or fewer defective widgets.
\item Evaluating both part (a) and part (b) numerically using Wolfram Alpha or your favorite software package, what is the relative error for your Poisson approximation\footnote{relative error is $|$true value - approximate value$|$ / true value}.
\end{enumerate}

\end{enumerate}

\end{document}

